\section{CAN-Ethernet-Gateway on Infineon AURIX TC297}
\label{sec:can-gateway}

This chapter focuses on the implementation of a CAN-Ethernet gateway.

\subsection{Hardware Overview: Infineon AURIX TC297}
% Describe the microcontroller and its relevant peripherals.
Placeholder for text...

\subsection{Software Implementation}
% Explain the software design and key algorithms.
Placeholder for text...

\subsubsection{C Code Example}
Here is an example of how to include a C code snippet.


\centering
\begin{sequencediagram}
    \newthread{runEthernetLogic}{runEthernetLogic}    
    \newinst{eth}{sendEthernetPacket}
    \newinst{dma}{getReceiveBuffer}
    \newinst{phy}{Phy\_Pef7071}

    \begin{call}{runEthernetLogic}{link()}{phy}{}
    \end{call}

    \begin{sdblock}{if g\_sendAliveMessage}{ }
        \begin{call}{runEthernetLogic}{"i am alive"}{eth}{}
        \end{call}
    \end{sdblock}

    \begin{call}{runEthernetLogic}{ }{dma}{rx\_data}
        \begin{sdblock}{if rx\_data}{ }
            \begin{call}{dma}{freeReceiveBuffer()}{dma}{}
            \end{call}
        \end{sdblock}
    \end{call}
\end{sequencediagram}
\caption{Ethernet Rx/Tx Logic}

\begin{figure}{!h}
\centering
\begin{sequencediagram}
    \newthread{app}{Application}
    \newinst{tx}{transmitCanMessage}
    \newinst{rx}{receiveCanMessage}
    \newinst{hw}{MultiCAN HW}

    \begin{call}{app}{message}{tx}{status}
        \begin{sdblock}{while busy}{ }
             \begin{call}{tx}{sendMessage()}{hw}{}
             \end{call}
        \end{sdblock}
    \end{call}

    \begin{call}{app}{ }{rx}{status, msg}
        \begin{call}{rx}{readMessage()}{hw}{}
        \end{call}
    \end{call}
\end{sequencediagram}
\caption{UML Diagramm: CAN Rx/Tx Logic}
\label{uml:can}
\end{figure}
\begin{figure}
\centering
\begin{sequencediagram}
    % Define lifelines
    \newthread{fwd_eth_can}{forwardEthernetToCan}
    \newinst{eth_driver}{Ethernet Driver}
    \newinst{can_driver}{MultiCAN Driver}

    % Sequence of calls with corrected syntax
    \begin{call}{fwd_eth_can}{receiveEthernetPacket}{eth_driver}{payloadLen}\end{call}

    \begin{sdblock}{alt}{[payloadLen > 0]}
        
        %\node[note, text width=4cm] (validation_note) [right=of fwd_eth_can] 
        %    {Internal Validation & Parsing:
        %    \begin{itemize}
        %        \item Check packet header and DLC
        %        \item Extract CAN ID, Data, frame type
        %    \end{itemize}
        %};
        
        \begin{sdblock}{alt}{[frame is DATA and ID is STANDARD]}
            
            \begin{callself}{fwd_eth_can}{IfxMultican\_Message\_init}{canMsg}\end{callself}

            % The call to the driver, which contains the loop
            \begin{call}{fwd_eth_can}{transmitCanMessage}{can_driver}{}
                
                \begin{sdblock}{loop}{[while TX buffer is busy]}
                    % The internal driver call
                    \begin{call}{can_driver}{sendMessage}{can_driver}{status}\end{call}
                \end{sdblock}
                
            \end{call}

        \end{sdblock}
    \end{sdblock}
\end{sequencediagram}
\\
\caption{UML Diagramm: Ethernet to CAN Message Forwarding}
\label{uml:ethtocan}
\end{figure}
\begin{figure}
\centering
\begin{sequencediagram}
    % Define lifelines
    \newthread{fwd_can_eth}{forwardCanToEthernet}
    \newinst{multican}{MultiCAN Driver}
    \newinst{eth}{Ethernet Driver}

    % Sequence of calls with corrected syntax
    \begin{call}{fwd_can_eth}{isRxPending}{multican}{isPending}\end{call}

    \begin{sdblock}{alt}{[isPending is TRUE]}
    
        \begin{call}{fwd_can_eth}{getPointer}{multican}{}\end{call}
        
        \begin{call}{fwd_can_eth}{readMessage}{multican}{rxMsg}\end{call}
        
        \begin{callself}{fwd_can_eth}{Construct UDP Payload from CAN data}{}\end{callself}

        \begin{call}{fwd_can_eth}{sendEthernetPacket}{eth}{}\end{call}
        
        \begin{call}{fwd_can_eth}{clearRxPending}{multican}{}\end{call}
        
    \end{sdblock}

\end{sequencediagram}
\caption{CAN to Ethernet Message Forwarding}
\end{figure}


\begin{lstlisting}[language=C, caption={Example of a simple CAN message sending function.}, label={lst:can_send}]
#include <stdio.h>

// Define CAN message structure
typedef struct {
    unsigned int id;
    unsigned char data[8];
    unsigned char dlc; // Data Length Code
} CAN_Message;

/*
 * @brief Sends a CAN message.
 * @param msg Pointer to the CAN_Message to be sent.
 */
void send_can_message(const CAN_Message* msg) {
    // Placeholder for actual hardware driver call
    printf("Sending CAN message with ID: 0x%X\n", msg->id);
    // ... implementation details for hardware registers ...
}

int main() {
    CAN_Message my_message;
    my_message.id = 0x123;
    my_message.dlc = 8;
    for (int i = 0; i < my_message.dlc; ++i) {
        my_message.data[i] = i;
    }
    
    send_can_message(&my_message);
    
    return 0;
}
\end{lstlisting}