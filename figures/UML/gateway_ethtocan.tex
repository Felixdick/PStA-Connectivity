\begin{figure}
\centering
\begin{sequencediagram}
    % Define lifelines
    \newthread{fwd_eth_can}{forwardEthernetToCan}
    \newinst{eth_driver}{Ethernet Driver}
    \newinst{can_driver}{MultiCAN Driver}

    % Sequence of calls with corrected syntax
    \begin{call}{fwd_eth_can}{receiveEthernetPacket}{eth_driver}{payloadLen}\end{call}

    \begin{sdblock}{alt}{[payloadLen > 0]}
        
        %\node[note, text width=4cm] (validation_note) [right=of fwd_eth_can] 
        %    {Internal Validation & Parsing:
        %    \begin{itemize}
        %        \item Check packet header and DLC
        %        \item Extract CAN ID, Data, frame type
        %    \end{itemize}
        %};
        
        \begin{sdblock}{alt}{[frame is DATA and ID is STANDARD]}
            
            \begin{callself}{fwd_eth_can}{IfxMultican\_Message\_init}{canMsg}\end{callself}

            % The call to the driver, which contains the loop
            \begin{call}{fwd_eth_can}{transmitCanMessage}{can_driver}{}
                
                \begin{sdblock}{loop}{[while TX buffer is busy]}
                    % The internal driver call
                    \begin{call}{can_driver}{sendMessage}{can_driver}{status}\end{call}
                \end{sdblock}
                
            \end{call}

        \end{sdblock}
    \end{sdblock}
\end{sequencediagram}
\caption{Ethernet to CAN Message Forwarding}
\end{figure}